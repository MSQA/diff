\documentclass[11pt]{article}

\usepackage{a4wide}
\usepackage[czech]{babel} 
\usepackage[utf8]{inputenc}
\usepackage[T1]{fontenc}
\usepackage{graphicx}
\usepackage{charter}
\usepackage{float}

\newcommand\uv[1]{\quotedblbase #1\textquotedblleft}%

\floatstyle{ruled}
\newfloat{vystup}{H}{lop}
\floatname{vystup}{Výstup}

\begin{document}
\author{David Marek}
\title{Programátorská dokumentace programu Diff}
\date{}
\maketitle{}

\newpage{}

\section{Úvod}

\paragraph{}
Unixový program {\tt diff} slouží k zobrazení rozdílů mezi dvěma soubory. 
Pro dva soubory řekne, které řádky byly přidány nebo naopak odebrány.
Program diff je proslulý na unixových systémech. Společně s 
programem {\tt patch} se staly programy, které napomohly open source vývoji.
Tento program je grafickou verzí programu diff. V přívětivém rozhraní
zobrazuje rozdíly mezi soubory a barevně je i zvýrazní. Samozřejmě umožňuje
i stejný výstup jako unixová utilita stejného názvu.
Tato dokumentace popisuje použité technologie, algoritmy a třídy použité v 
programu.

\section{Technologie}
            
\paragraph{}
Program je napsán v programovacím jazyce Java, pro grafické rozhraní byl použit
toolkit Swing. Program je díky tomu multiplatformní. Bude fungovat všude tam, 
kde je nainstalována Java (což v dnešní době může být kterýkoli operační 
systém).

\section{Použitý algoritmus}

\paragraph{}
Rozdíl dvou souborů je typický příklad pro použití metody dynamického 
programování. Algoritmus se nazývá \emph{Longest Common Subsequence}, tedy 
hledání nejdelší společné podposloupnosti, což se může zdát podivné, hledat
nejdelší společnou podposloupnost, když chceme znát rozdíly mezi oběma soubory.
Avšak když vezmeme největší společnou podposloupnost, pak vše ostatní co zbyde
jsou námi hledané změny. K algoritmu si tady dovolím říct jen tolik, že 
porovnává v podstatě každý řádek s každým, kvůli tomu běží v čase $O(n^2)$.
Komentáře v tříde {\tt LongestCommonSubsequence} tento algoritmus dopodrobna
popisují a společně se samotným kódem tvoří tu nejlepší dokumentaci. 

V tomto algoritmu existuje spousta míst, kde lze udělat tu menší, tu větší 
optimalizace. První optimalizace vyplývá ze statistiky, většina změn bývá 
uprostřed, soubory mívají většinou společný začátek a konec. Proto ještě před
samotným počítáním nejdelší společné podposloupnosti můžeme odebrat společné 
řádky ze začátku a konce obou soubourů. 

Další optimalizace tkví v tom, že algoritmus porovnává každý řádek s každým,
tedy porovnává řetězce. Pokud se vytvoří mapování řetězců na čísla tak se 
ušetří spousta porovnání. 

První optimalizace byla implementována, druhá byla odkládána, až se nakonec do
finální verze nestihla dostat.

\section{Vytváření diffu}

\paragraph{}
K vytváření výstupu slouží celá hierarchie tříd odvozených od třídy 
{\tt CreateDiff}. Třída {\tt CreateDiff} vytváří pouze seznam objektů 
{\tt SequenceElement}, což jsou řádky s příznakem, zda byly přidány, odebrány 
nebo nezměněny. Od třídy {\tt CreateDiff} potom dědí další třídy 
{\tt CreateNormalDiff} a {\tt CreateUnifiedDiff}, které vytváří smysluplnější 
výstup do souboru.

\section{Grafické rozhraní}
Grafické rozhraní je napsáno v toolkitu Swing. Hlavní okno je reprezentován 
třídou {\tt Window}. Většinu kódu této třídy vygenerovaly NetBeans. Pak už je 
v ní jen spousta funkcí obsluhujících jednotlivé události.

Změny jsou zobrazovány ve widgetu JTable, v prvním sloupci orientační 
číslování, ve druhém řádku souboru podbarvené podle změn. Toto rozvržení je 
vytvářeno v modelu tabulky ve třídě {\tt DiffTableModel}. Dále bylo potřeba
pro podbarvování řádků kódu v tabulce vytvořit třídu
{\tt DiffLineCellRenderer}, která přepisuje třídu 
{\tt DefaultTableCellRenderer}.

\end{document}

